
\section{Language Features}
\label{sec:features}
% 5. Describe the features your language provides and explain why this set of
% features is appropriate.
The current iteration of DeckBuild gives a game designer the
following features:

\begin{itemize}
\item Declarative enumeration of the content of a card comprising its
      \emph{name}, card \emph{type},
      what the card does when played (\emph{effects}), and the \emph{cost}
      of the card in order to buy it.
\item Declarative enumeration of the parameters to the mechanics of a turn in
      the game, namely the number
      of \emph{actions} and \emph{buys} a player starts with on each turn,
      the number of cards \emph{discarded} at the end of a turn, and the
      number of cards \emph{drawn} in-between turns.
\end{itemize}

These features give game designers a way to quickly express new cards and
game mechanics. This speed of prototyping new deckbuilding game variants
facilitates the exploration process. Namely, a deckbuilding game designer
will likely have a general idea what the cards should do, and have a
feeling that they will interact well together. Coming up with specific
numbers to put on the cards however requires \emph{exploring} the state
space of possible cards.

As discussed earlier in DeckBuild's {\bf Goals} (Section \ref{sec:goals}),
this process is indicative of what DeckBuild should give a game designer.
Although this set of features leaves much to be desired (see Section
\ref{sec:future} for full discussion), we believe it to be appropriate
for our goals. See the last page of this document for a complete grammar
for DeckBuild.

% TODO: figure out how to get RHS grammar indenting to work properly...
%\subsection{Grammar}
%\begin{footnotesize}
%\begin{grammar}
%\input{../grammar/grammar.tex}
%\end{grammar}
%\end{footnotesize}

\subsection{Example Program}
% 6. Show at least one sample program in your language and explain what it
% does.

\inputminted
[ %frame=lines
, framesep=2mm
, fontsize=\small %\tiny
%, linenos
] {haskell}{BaseQuote.hs}

Above is a complete example program. The program defines ten cards and
one turn mechanics declaration. The ten cards are parsed and then compiled
into a Haskell list \mintinline{haskell}{kingdomCards :: Card}.

\section{Implementation}
\subsection{QuasiQuoter \& Compiler}
% 7. Describe how you implemented the language.

\subsection{Runtime System}
% 8. Explain what if any services are provided in a “run-time system.”



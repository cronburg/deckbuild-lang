
\section{Language Features}
\label{sec:features}
% 5. Describe the features your language provides and explain why this set of
% features is appropriate.
The current iteration of DeckBuild gives a game designer the
following features:

\begin{itemize}
\item Declarative enumeration of the content of a card comprising its
      \emph{name}, card \emph{type},
      what the card does when played (\emph{effects}), and the \emph{cost}
      of the card in order to buy it.
\item Declarative enumeration of the parameters to the mechanics of a turn in
      the game, namely the number
      of \emph{actions} and \emph{buys} a player starts with on each turn,
      the number of cards \emph{discarded} at the end of a turn, and the
      number of cards \emph{drawn} in-between turns.
\end{itemize}

These features give game designers a way to quickly express new cards and
game mechanics. This speed of prototyping new deckbuilding game variants
facilitates the exploration process. Namely, a deckbuilding game designer
will likely have a general idea what the cards should do, and have a
feeling that they will interact well together. Coming up with specific
numbers to put on the cards however requires \emph{exploring} the state
space of possible cards.

This process is indicative of what DeckBuild should give a game designer.
Although this set of features leaves much to be desired (see Section
\ref{sec:future} for full discussion), we believe it to be appropriate
for our goals.

See the last appendix for a complete grammar for DeckBuild.

% TODO: figure out how to get RHS grammar indenting to work properly...
%\subsection{Grammar}
%\begin{footnotesize}
%\begin{grammar}
%% -----------------------------------------------------------------------------
% - Top-level decls:

<deckDecls>  ::= <deckDecl>*

<deckDecl>   ::= <cardDecl> | <turnDecl>

% -----------------------------------------------------------------------------
% - Card declaration grammar rules:

<cardDecl>    ::= "card" <cardID> "::" <cardType> "{" <cardDescr> "}" "costs" <intLit>

<cardDescr>   ::= <effectDescr>* <englishDescr>

<englishDescr> ::= <stringLit>*

<effectDescr>  ::= <plusOrMinus><intLit> <effectType>

<plusOrMinus> ::= "+" | "-" 

% -----------------------------------------------------------------------------
% - Phase / rulebook declaration grammar rules:
% -   TODO: custom named phases (future), rulebook for custom rules other than turns

<turnDecl>   ::= "turn" <turnID> "{" <phaseDescr>* "}" 

<phaseDescr> ::= <phaseName> <phaseInt>

<phaseInt>   ::= <intLit> | "all"

% looks like: phases {Action 1 Buy 1 Discard all Draw 5} for standard rules.

% -----------------------------------------------------------------------------
% - Keyword and built-in-to-parsec grammar rules:
<effectType>  ::= "actions"   | "coins"  | "buys" | "cards" | "victory"

<cardType>    ::= "Treasure" | "Action" | "Victory"

<phaseName>   ::= "action"   | "buy"    | "discard" | "draw"

<*ID>         ::= <identifier>

<*Lit>      ::= As defined in Text.Parsec.Token

<identifier>  ::= As defined in Text.Parsec.Token

% Note that we don't actually need to specify the type as that is implied by the
% action
% Maybe add these back? Not sure if we need these. - Raoul
%
% <charLit>    ::= As defined in Text.Parsec.Token
%
% <stringLit>  ::= As defined in Text.Parsec.Token
%
% <identifier> ::= As defined in Text.Parsec.Token
%
% <whiteSpace> ::= As defined in Text.Parsec.Token

%\end{grammar}
%\end{footnotesize}

\subsection{Example Program}
% 6. Show at least one sample program in your language and explain what it
% does.

\inputminted
[ %frame=lines
, framesep=2mm
, fontsize=\footnotesize %\small %\tiny
%, linenos
] {haskell}{BaseQuote.hs}

Above is a complete example program. The program defines ten cards and
one turn mechanics declaration. The ten cards are parsed and then compiled
into a Haskell function \mintinline{haskell}{kingdomCards :: [RuntimeCard]}.
The ten cards are also parsed into an enumeration data type where,
for example, \mintinline{haskell}{VILLAGE :: CardName}. This in particular
facilitates the use of host-language features of Haskell when designing
cards with decidedly non-declarative effects (see
Appendix \ref{appendix:complexcard} ) for an example).
The turn declarations are parsed and compiled in the same manner as the
card declarations, except they are compiled into a Haskell function
\mintinline{haskell}{turnRules :: [Turn]}.

% 7. Describe how you implemented the language.
The compilation just described is performed using Haskell's quasiquotation
facilities. As a result, the recommended programming pattern is to define
a single \mintinline{haskell}{[deck| ... |]} quasiquoted program at the
beginning of a Haskell module. The Haskell module is named based on why the cards in
the module are being group together. For example, a deckbuilding game based
on World War II might have a module named `WWII' containing cards named
after airplanes used in World War II.

With a complete deck of cards defined in a module, the runtime system now
consumes said module. Presently the runtime system can be interfaced with
in three ways:

\begin{itemize}
\item As a Haskell backend API for performing probabilistic inference
      on the possible states of the game given a model for the play
      heuristics of two computer-players playing the game. This is where
      the Machine Learning users (with Haskell experience) interested
      in deckbuilding games can use DeckBuild.
\item As a web socket API for designing web-based deckbuilding games.
      This is where web programmers (with minimal to no Haskell experience)
      can make use of DeckBuild to create a deck build game and corresponding
      web site tailored to their genre of interest.
\item As a command-line interface for humans to play against one another
      (making use of the web sockets API). This is where a game designer can
      have human subjects play-test their game without having to print out
      mock-ups of cards.
\end{itemize}

% 8. Explain what if any services are provided in a “run-time system.”
The runtime system itself is implemented as Haskell cabal package separate
from (but referencing / importing) the DeckBuild language package. It is
implemented in the state monad for maintaining a stateful view of the game
as turns progress in the game. We also use the IO monad for executing
player-based events in the game. The core functionality of the runtime system
handles functionality common to all deckbuilding games comprising:

\begin{itemize}
\item Maintaining piles of cards
  \begin{itemize}
  \item Moving cards between decks (enforcing no `card-duplication-bugs')
  \item 
  \end{itemize}
\item Library of:
  \begin{itemize}
  \item Getters and setters for \mintinline{haskell}{Game} state
  \item Execution of common events (draw, shuffle, discard, ...)
  \item Common \mintinline{haskell}{Game} state questions:
    \begin{itemize}
    \item How many cards of type `X' are in my hand?
    \item Can I currently buy card `Y'?
    \item Can I currently play card `Z'?
    \item Has the game ended? If so, who won?
    \end{itemize}
  \end{itemize}
\end{itemize}

The use of Haskell as the host-language of DeckBuild gave the runtime system
a lot of power in being able to make guarantees about the state of a game.
In particular the first-class nature of IO lets the runtime system divide
up the portions of the system which are and are not allowed to perform IO.
This division makes IO bugs impossible to create when designing and writing
purely functional components of the runtime system. Similarly, the immutable
nature of Haskell data instances makes shared-memory bugs impossible. This is
important in DeckBuild's runtime system when communication occurs between the
core components of the runtime system (functions inside the state monad)
and the computer-player \& web socket components (and other components not
allowed to directly alter the state of the game).

In the current iteration of our language and the runtime system, some kinds
of concepts such as \mintinline{haskell}{ACTIONS} and \mintinline{haskell}{COINS}
are hard-coded into the runtime system and as concrete syntax in the language.
These names are not inherently common to all deckbuilding games, and will be
abstracted out in a future iteration of the DeckBuild language. See Section
\ref{sec:future} for a discussion of how this and other parts of the runtime
system and language will be given better abstractions in the future.

\subsection{Runtime System}



%\onecolumngrid
\clearpage
\onecolumn
\appendix

\section{Example Program - Haskell Output}
\inputminted
[ frame=lines
, fontsize=\small %\fontsize{1mm}{1mm} %\tiny
, framesep=2mm
%, linenos
] {haskell}{../pres/QuoteOutput.hs}

\section{Example Program - Complex Card Effects}

\inputminted
[ frame=lines
, framesep=2mm
, fontsize=\small %\fontsize{1mm}{1mm} %\tiny
, linenos
] {haskell}{../pres/ComplexEffects.hs}

The code above defines the state transformation to be performed on the game when
a \mintinline{haskell}{CELLAR} is played. The state monad is required because
the effect moves cards around between piles in the state of the game. The IO monad
is also required because the \mintinline{haskell}{CELLAR} effect requires asking
the player to pick cards. This example also illustrates the use of:
\begin{itemize}
\item \mintinline{haskell}{cards   :: Pile   -> [CardName]}, 
\item \mintinline{haskell}{hand    :: Player -> Pile}
\item \mintinline{haskell}{p1      :: Game   -> Player}
\item \mintinline{haskell}{mayPick :: Game   -> CardName -> IO (Maybe CardName)}
\end{itemize}
These are the kinds of operations DeckBuild will support natively in the
future when it has support for function declarations and expressions.

\section{Runtime System - State Format}
\begin{small}
$\lambda$\verb|> runGreedy (0.5, 0.5)|
\end{small}
\begin{Verbatim}[fontsize=\small]
Player1:
    name   = "Greedy1"
    hand   = [ESTATE, GOLD, PROVINCE, PROVINCE, SILVER]
    inPlay = []
    deck   = [SILVER,   PROVINCE, COPPER, SILVER,  COPPER
             ,ESTATE,   COPPER,   COPPER, VILLAGE, VILLAGE
             ,PROVINCE, ESTATE,   SILVER, COPPER]
    dscrd  = [SILVER, SILVER, SILVER, COPPER, COPPER, PROVINCE]
    buys=1, actions=1, money=0

Player2:
    name   = "Greedy2"
    hand   = [COPPER, COPPER, COPPER, COPPER, VILLAGE]
    inPlay = []
    deck   = [SILVER,   SILVER, GOLD, COPPER, COPPER,   ESTATE
             ,GOLD,     ESTATE, GOLD, ESTATE, PROVINCE, VILLAGE
             ,PROVINCE, GOLD]
    dscrd  = [SILVER, COPPER, SILVER, SILVER, SILVER, PROVINCE]
    buys=1, actions=1, money=0

Trash: []
Supply: [(COPPER,60),  (CELLAR,10),  (MOAT,10),       (ESTATE,8)
        ,(SILVER,27),  (VILLAGE,6),  (WOODCUTTER,10), (WORKSHOP,10)
        ,(MILITIA,10), (REMODEL,10), (SMITHY,10),     (MARKET,10)
        ,(MINE,10),    (DUCHY,8),    (GOLD,25),       (PROVINCE,0)]
Turn #: 30
\end{Verbatim}

Above is a snapshot of the state of the runtime system after two
probabilistic computer-player models played a game against each other.
The player models are given direct access to this game-state information
on each turn of the game. In future iterations of the runtime system
their will likely be a feature whereby the players can `remember' what
they did on previous turns.

In the snapshot, the \mintinline{haskell}{Supply :: [(CardName,Int)]} where
the \mintinline{haskell}{Int} corresponds to the number of copies of
that particular card in the supply. The runtime system is implemented in
this manner so as to reduce the space-complexity constant. In a future
iteration of the runtime system, game designers will be
able to turn on and off these kinds of optimizations on a
per-\mintinline{haskell}{Pile} basis.

Similarly, a game designer may wish to specify the order in which cards
appear in a \mintinline{haskell}{Pile}. In the current implementation
of the runtime system, ordering of most piles defaults to LIFO queue
order (except a player's hand which is shown alphabetically).


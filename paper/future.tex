
\section{Future Work}
\label{sec:future}
% 10. Discuss what you would do next on your language if you were going to
%     continue working on it.

We would like to add a number of features to the language. First, we would like
to add syntax for more complicated effects. Currently, we only support action
and buy effects. The original game of Dominion allowed for more complex actions
such as drawing multiple cards and attacking other players. This would
complicate the language but is necessary to to capture the essence of the
original game.

We would also like to develop a type system that models the cards and the rules
of the game. This would lead to development of a type-checker that we envision
would help designers catch game design errors earlier.

We also plan to tweak the language with respect to how much is built into the
language and how much is made generic. For example, \mintinline{haskell}{COINS}
are DeckBuild primitives, but the user may be designing a game that doesn't use
or need money. One possible solution is to provide common game artifacts such as
money as libraries that use more generic primitives. For example, we may have a
\mintinline{haskell}{RESOURCE} language primitive that may be configured to give
the same effect as \mintinline{haskell}{COINS}.

We would also like to have more relevant and useful debugging information.
Currently debugging information is limited to line numbers in the source file.
This would not be very useful to non-Haskell programmers.

For a better game design experience, we envision an automated card balancer that
attempts to balance the different design axes. While this isn't related to the
language, we think this would be a useful tool that could drive greater adoption
of DeckBuild.

The runtime system would ultimately be the tipping point for designers. Even if
game designers could easily specify the cards and game mechanics in DeckBuild,
they would have no incentive to do so if they had to implement the runtime
system too. To start with, we would like to have graphical client-server
support, markup language conversion tools, and a library of AI players that
designers can choose from.

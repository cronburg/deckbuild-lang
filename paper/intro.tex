
% This is a possible breakdown of sections. Feel free to re-arrange / rename
% things, but try to at least keep the labels (I reference 'sec:intro'
% and 'sec:goals' later in the paper)

\section{Introduction - Target Domain \& Users} \label{sec:intro}
\subsection{Domain}
Like most table top or board games, deck building card games do not have many automated systems or
testing tools for their general use or creation. Only the most popular games in the genre have software that
either simulates game play or running strategies. Typically the software is closed or proprietary, and
making changes to the software to fit personal needs or a new game is discouraged. We have identified a need
to facilitate the creation and testing of new deck building card games.
\subsection{Users}
The intended audience for DeckBuild is card game makers who want
an easy way to describe and test new cards, card types, or card game rules. Specifically, DeckBuild
supports describing cards and rules from the sub-genre of deck building card games such as Dominion or Magic the
Gathering. The card game makers we expect to adopt DeckBuild either have a programming
background or will be working closely with a programmer.
\\
In its current iteration, the DeckBuild
suite provides an easy way to represent cards, consumable by the runtime system library.
The technical designer (programming partner) can then connect the functionality of the cards to
the runtime library in any way they wish. The programmer does this using DeckBuild's provided
artifacts, such as the game client to test the cards and the computer-players with custom heuristics.
While we hope to make those tools work without manipulating the
runtime library behind the scenes to lower the barrier of entry to our language, if the card game maker
wishes to assess new functionality in their cards and game rules through these artifacts, then
they should be prepared to provide some implementation of their own. Card game makers who wish to
extend only the grammar of existing poplar deck building games will find all that functionality
available in DeckBuild.

\subsection{Existing Languages}
Presently there are no languages suited to describing features of deck building card games in general
terms. Existing software such as Dominion Online\cite{DomOnline} or Magic the Gathering Online \cite{MTGOnline}
allows for playing officially licensed games with people or AIs, but does not open up
any sort of design API / interface. The closest competitor to DeckBuild is Dominiate, an
open source project by Robert Speer. It is a Dominion Simulator written in coffeescript.\cite{Dominiate}
\\
Dominiate provides a way for end users to specify policies of a Dominion AI in terms of:
\begin{itemize}
\item Priority to buy cards in.
\item Priority to play cards in.
\item Other heuristics available to the player at any game state.
\end{itemize}
\\
While DeckBuild and Dominiate share a common goal of letting users evaluate
strategies of given combinations of cards, Dominiate does not include support for new cards.
Logic for cards and rules is spread across its runtime system making it
difficult to add functionality for new cards. Dominiate can then be seen not as a language competitor
for game makers to compare, but rather a runtime system that DeckBuild would hope to leverage in the
future as an output format. Specifically, it would benefit both communities to be able to specify
cards and rules in DeckBuild, but output cofeescript source for use in the Dominiate simulator.
\\
\begin{figure}
\inputminted
[ %frame=lines
, framesep=2mm
, fontsize=\footnotesize %\small %\tiny
%, linenos
] {cfs}{SimpleMoney.coffee}
\caption{An example heuristic written in Dominiate}
\end{figure}
\section{Goals} \label{sec:goals}
DeckBuild is at its core a way for card game makers to describe cards
for use by general purpose programs or tools. Our goals reflect this,
with more focus on end-user features than performance or optimization
features.

\subsection{Readability}
Currently two representations for deck building cards are accepted by our target communities. The game maker
community represents the card as a physical artifact with plain English functional descriptions
of what the card can do. This can result in ambiguous and difficult to parse descriptions. The
programmer making tools for these games represents cards in terms of their functionality (how they affect
the game state). While providing concrete benefits (able to be parsed), in general purpose languages this
requires the programmer to design deck building abstractions of their own. A programmer needs such abstractions
if they wish to centralize the card content in a single file. Centralizing this content improves readability
by requiring the reader to go to one place to see exactly how a particular card interacts with the game engine.
The card and rule abstractions in DeckBuild provide a functional (as opposed to imperative) interface to
the artifacts DeckBuild provides. This interface is designed with human readability in mind, making
compromises in terms of expressive power. Non-programmers can read DeckBuild's card format as if they were
holding a physical card on paper.

\subsection{Consumability}
In implementation, programmers wish to describe cards and rules by how they affect the game
state. What these cards do and how they are used can widely vary. Since functional correctness
is important to game designers and programmers, having DeckBuild automatically guarantee
it is also important. If a new card requires a specific resource of the game
state or runtime, it should be obvious upon describing the card.
This could be done through object-oriented interfaces or other contractual classification
of card objects, but that can quickly become difficult to manage, especially for a large number of cards. We hypothesize that
by representing cards and rules in a uniform syntax and parsing them in Haskell, we can easily provide type safety and
card requirements to any runtime system that wants to consume DeckBuild's representation.
\subsection{Support and Resources}
The DeckBuild language can better benefit its early adopters if they have fun tools to play with out-of-the-box to test their
new cards and game rules with. We hope to achieve, for the programmers working
with these card game makers, an easy way to take the standard game representation and integrate it in
programs. DeckBuild provides a game state format and runtime system with a basic AI library and client interfaces.
These APIs and features allow for automatic consumption of card descriptions. DeckBuild also provides an interface
for specifying new runtime system functionality in the host language (Haskell). Eventually, it would fully
realize our goal if the programmer could provide the functionality that corresponded to abstract representation of arbitrary
English card descriptions that DeckBuild parses, and then the language automatically makes those connections. For example, if a new
simulator were to be released that takes card representations in a novel way, the programmer should be able to simply write the artifact
generator from our abstract representation to that new format. This way both the programmer and the card game maker gain the benefit
of interoperability.
\subsection{Metrics and Measurement}
Since the goals of DeckBuild are user-centric, the successfulness of our language can be measured using the quantified
amount of effort it takes for the programmer to connect the output of the defined cards and rules to the runtime implementation of their
choice. Reducing this effort can come about from meeting any of the following goals:
\begin{itemize}
\item Making the code readable, yet unambiguous
lets the programmer reuse functionality since the size of the grammar they must support is smaller.
\item Making the internal representation of
the card type safe and consumable reduces the amount of checks the programmer must perform before trying to integrate the functionality of
the card with a new system.
\item Through good tool support and out-of-the-box functionality, we hope to abstract away most of the common
boiler plate the programmer would want to write such as a generic card game simulator or client-server system.
\end{itemize}
\\
For this initial implementation, we also measure the effectiveness in terms of the expressiveness that the card game designer gets
without having to delegate tasks to their programmer. The card game designer should be able to use DeckBuild to give a non-ambiguous
description of what she wants without necessarily having to tell the programmer in English.
DeckBuild is therefore a success if one is able to describe any arbitrary cards from the
official Dominion set. We hope to show that these aspects of our language are complete while still remaining robust enough to describe any
new card. The observable metric of how robust DeckBuild is can be then seen as how well we minimize the amount of code the programmer has to
write in using the runtime library when providing functionality for new cards.



% Describe the features your language provides and explain why this set of
% features is appropriate.
\begin{frame} \frametitle{Language Features}
\begin{itemize}
\item Can declaratively enumerate 
  \begin{itemize}
  \item Cards    - name, type, effects, cost
  \item Rulesets - e.g. how many cards you draw per turn
  \end{itemize}
\item Can define complex card effects directly from Haskell
\end{itemize}
\end{frame}

% Show at least one sample program in your language and explain what it does.
\begin{frame}[fragile=singleslide] \frametitle{Example Program - Input}
\begin{columns}
  % FIRST COLUMN:
  \column{.5\linewidth}
    \begin{minted}
    [ frame=lines
    , framesep=2mm
    , fontsize=\tiny
%    , linenos
    ] {haskell}
[deck|
card Cellar  :: Action {
  +1 actions
  "Discard any number of cards."
  " +1 Card per card discarded"
} costs 2

card Chapel  :: Action {
  "Trash up to 4 cards from your hand"
} costs 2

card Village :: Action {
  +1 cards
  +2 actions
} costs 3

card Woodcutter :: Action {
  +1 buys
  +2 coins
} costs 3

card Copper :: Treasure {
  +1 coins
} costs 0

card Silver :: Treasure {
  +2 coins
} costs 3
    \end{minted}

  % SECOND COLUMN:
  \column{.5\linewidth}
    \begin{minted}
    [ frame=lines
    , framesep=2mm
    , fontsize=\tiny
%    , linenos
    ] {haskell}
card Gold :: Treasure {
  +3 coins
} costs 6

card Estate :: Victory {
  +1 victory
} costs 2

card Duchy :: Victory {
  +3 victory
} costs 5

card Province :: Victory {
  +6 victory
} costs 8

turn Dominion_Standard {
  action 1
  buy 1
  discard all
  draw 5
}





|]
    \end{minted}
\end{columns}
\end{frame}

\begin{frame}[fragile=singleslide] \frametitle{Example Program - Quasiquoter \& Code Generator}

\inputminted
[ frame=lines
, framesep=2mm
, fontsize=\fontsize{1mm}{1mm} %\tiny
, linenos
] {haskell}{CodeGen.hs}

\end{frame}

\begin{frame} \frametitle{Example Program - CodeGen Output}
\inputminted
[ frame=lines
, framesep=2mm
, fontsize=\fontsize{1mm}{1mm} %\tiny
, linenos
] {haskell}{QuoteOutput.hs}
\end{frame}

\begin{frame}[fragile=singleslide] \frametitle{Example Program - Complex Card Effects}
\begin{columns}
  \column{.5\linewidth}
    \begin{itemize}
    \item Can reference quasiquoted EDSL code from Haskell
    \item Future: design imperative-friendly (non-Haskell) EDSL for specifying
          complex card effects
    \item Example below - implementation of \verb|CELLAR| effects
    \end{itemize}
  \column{.5\linewidth} \begin{center} \includegraphics[width=.4\columnwidth]{cellar.jpg} \end{center}
\end{columns}
\inputminted
[ frame=lines
, framesep=2mm
, fontsize=\fontsize{1mm}{1mm} %\tiny
, linenos
] {haskell}{ComplexEffects.hs}
\footnotetext[1]{\tiny card image \& content by D. X. Vaccarino}
\end{frame}


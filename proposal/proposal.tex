\documentclass{acm_proc_article-sp}
\usepackage{graphicx}
\usepackage[abbrev]{amsrefs}

% Turn off space for copyright notice.
\makeatletter
\renewcommand{\@copyrightspace}{}
\makeatother

\begin{document}

\title{EDSL Proposal: Deck-Building Card Games}

\numberofauthors{1}
\author{
  \alignauthor
  Matthew Ahrens \& Karl Cronburg \& Raoul Veroy \\
  \affaddr{Tufts University} \\
  \affaddr{Medford, MA}       \\
  \email{\{mahrens,karl,rveroy\}@cs.tufts.edu}
}
\date{Oct. 13, 2014}

\maketitle

\begin{abstract}
Abstract.
\end{abstract}

\section{Domain \& Motivation}
\label{sec:domain}
% Questions (1) and (2)
The domain of our language is deck-building card game design. In particular we
plan to focus on the deck-building terminology and basic rule infrastructure
used in the game \emph{Dominion} \cite{Vaccarino2008}. We further hope to,
in future iterations of our DSL, abstract out the specifics of \emph{Dominion}
resulting in a domain of fully generic deck-building.

The intended users of our DSL are deck-building game designers. The designers
of such games usually posess artistic skills, so we plan to make the card
design process as declarative as possible with minimal programming experience
necessary. However, when more complex card features / abilities are
desired, the designer will be given direct access to the Haskell run-time
system of the game.

Such access to the runtime system allows experienced
programmers to leverage existing general purpose language features they
are familiar with, while retaining the declarative nature of how cards are
defined. For less experienced programmers, example implementations of
existing card features will be made available.

While this ultimately necessitates a rudimentary knowledge of Haskell
to design non-trivial game features, it also hides the monadic nature of
the deck-building runtime system.

\section{Existing Tools}
\label{sec:existing_tools}
% Question (4)
As just mentioned in Section \ref{sec:domain}, the primary advantage of our
plan for a semi-transparent runtime system \footnotemark[1] is that game
designers have direct access to an existing general purpose programming
language while hiding the
monadic implementation of game state transitions. Designers simply specify
the abilities of a card in a functional manner, and the runtime system does
the rest.

\footnotetext[1]{The runtime system is transparent in that it gives
the designer direct access to state variables such as the decks on
the table and the hands of cards each player is holding. It is
non-transparent in that the monadic Haskell implementation of state
transitions will be hidden from the designer.}

In contrast, existing systems for card games in general comprise general
purpose programming languages. Such languages require considerable
programming experience. As a result the general trend is for game designers
to design games by hand, then later employ experienced programmers to
bring their creations into the digital world. Such a process is ill-suited
to prototyping and actively developing a deck-building game because the
designer is not intimately involved in the digitization process.

Furthermore, when we take into account the small age
of deck-building games (with \emph{Dominion} arguably
being the classic seminal deck-building game, dating back to only 2008)
it is no wonder the automated tools and languages for designing and
prototyping such games simply do not exist.

\section{Goals}
\label{sec:goals}
% Question (3)
% Question (8)

\section{Core Features}
\label{sec:features}
\subsection{Language}
\label{sec:language}
% Question (5)

\subsection{Run-Time System}
\label{sec:runtime_system}
% Question (6)

\section{Use Cases}
\label{sec:use_cases}
% Question (9)

\section{Haskell}
\label{sec:haskell}
% Question (7)

\section{Evaluation Criteria}
\label{sec:evaluation}
% Question (10)

\begin{biblist}
\end{biblist}

%\balancecolumns

\end{document}

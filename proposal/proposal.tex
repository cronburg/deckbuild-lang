\documentclass{acm_proc_article-sp}
\usepackage{graphicx}
\usepackage[abbrev]{amsrefs}

% Turn off space for copyright notice.
\makeatletter
\renewcommand{\@copyrightspace}{}
\makeatother

\begin{document}

\title{EDSL Proposal: Deck-Building Card Games}

\numberofauthors{1}
\author{
  \alignauthor
  Matthew Ahrens \& Karl Cronburg \& Raoul Veroy \\
  \affaddr{Tufts University} \\
  \affaddr{Medford, MA}       \\
  \email{\{mahrens,karl,rveroy\}@cs.tufts.edu}
}
\date{Oct. 13, 2014}

\maketitle

\begin{abstract}
Abstract.
\end{abstract}

\section{Domain \& Motivation}
\label{sec:domain}
% Questions (1) and (2)
%   What is the domain of your language?
%
%   Who are the intended users? What skills do they have and what kind of
%   programming tasks do they need to accomplish?

The domain of our language is deck-building card game design. In particular we
plan to focus on the deck-building terminology and basic rule infrastructure
used in the game \emph{Dominion} \cite{Vaccarino2008}. We further hope to,
in future iterations of our DSL, abstract out the specifics of \emph{Dominion}
resulting in a domain of fully generic deck-building.

The intended users of our DSL are deck-building game designers. The designers
of such games usually posess artistic skills, so we plan to make the card
design process as declarative as possible with minimal programming experience
necessary. However, when more complex card features / abilities are
desired, the designer will be given direct access to the Haskell run-time
system of the game.

Such access to the runtime system allows experienced
programmers to leverage existing general purpose language features they
are familiar with, while retaining the declarative nature of how cards are
defined. For less experienced programmers, example implementations of
existing card features will be made available.

While this ultimately necessitates a rudimentary knowledge of Haskell
to design non-trivial game features, it also hides the monadic nature of
the deck-building runtime system.

\section{Existing Tools}
\label{sec:existing_tools}
% Question (4)
%   Why are existing languages ill-suited to the task?

As just mentioned in Section \ref{sec:domain}, the primary advantage of our
plan for a semi-transparent runtime system \footnotemark[1] is that game
designers have direct access to an existing general purpose programming
language while hiding the
monadic implementation of game state transitions. Designers simply specify
the abilities of a card in a functional manner, and the runtime system does
the rest.

\footnotetext[1]{The runtime system is transparent in that it gives
the designer direct access to state variables such as the decks on
the table and the hands of cards each player is holding. It is
non-transparent in that the monadic Haskell implementation of state
transitions will be hidden from the designer.}

In contrast, existing systems for card games in general comprise general
purpose programming languages. Such languages require considerable
programming experience. As a result the general trend is for game designers
to design games by hand, then later employ experienced programmers to
bring their creations into the digital world. Such a process is ill-suited
to prototyping and actively developing a deck-building game because the
designer is not intimately involved in the digitization process.

Furthermore, when we take into account the small age
of deck-building games (with \emph{Dominion} arguably
being the classic seminal deck-building game, dating back to only 2008)
it is no wonder the automated tools and languages for designing and
prototyping such games simply do not exist.

\section{Goals}
\label{sec:goals}
% Question (3)
%   What goals are you trying to accomplish with your design? (For example, the
%   designers of ESP wanted to make it easier to develop correct, modular code for
%   firmware while not paying too much of a performance cost.)


Physical cards are typically represented in plain english with an ambiguous ruleset 
for how they react to other cards. Our grammar will encompass the common elements from 
deck building card games with the primary goal to let game designers prototype new cards
easily in a lexicon they are comfortable with. The usable product delivered after the 
designer uses our language will be a method of prototyping their new cards and rulesets
digitally before they physically make cards. This will be achieved through generating a 
server instance of the game with an api specification, and a static client that can consume
any specification made from cards. By using our language, the designer only has to focus on
designing the rules and relationships between cards in order to generate a playable product.
The intention is for game designers to quickly iterate over ideas for cards, and test them 
without writing extensive logic, boilerplate for UI, or physically printing cards in order to 
get people to playtest.

% Question (8)
%   Explain what benefits domain experts will get from using your language. For example, do they get
%   multiple artifacts from a single program? Can they write substantially less code? Is the code more
%   likely to be correct or efficient? Can specialized analyses be applied?

Game designers benefit from using our language by being able to define the actions a card
can perform in the way it would be printed to the player. This is beneficial so that the designer
can know that the card will behave in a specific way when played with other cards, and it is good for
the player since unifying the grammar will make the game easier to understand. By generating a server
from the code written in our language, the game designer does not have to worry about programmatically 
enforcing the rules they have composed. The automatic generation of a consumable specification for the 
game server allows generic clients -- such as the one we are going to provide -- to be made in the environment
and general programming language of choice. If the designer would like a more specialized client for their game,
then they can simply write the card definitions in our language and have a more experienced programmer write
only the specialized client in their general purpose language of choice.


\subsection{Language}
\label{sec:language}
% Question (5)
%   Describe what features your language will provide and give some rationale
%   for why this set of features is appropriate.


\subsection{Run-Time System Artifacts \& Use Cases}
\label{sec:runtime_system}
% Question (6) and (9)
%    Explain what if any services you will need to provide in a "run-time system"
%
%    Develop three to four use cases, each of which specifies:
%    a. The specific task the user is trying to accomplish
%    b. The information the user would have to supply, perhaps in the form of preliminary syntax for
%       your language
%    c. A high-level description of the code you intend to generate from the supplied information. If you
%       plan to generate multiple artifacts from a single description, you should describe each of them.
%    d. A brief discussion of why you will have sufficient information to generate the desired
%       implementation from the user-supplied information.

As described in Section \ref{sec:goals}, we have various goals for our
language. Enumerating these goals as specific run-time system use cases
and corresponding artifacts:

\begin{enumerate}
% Use case #1 - unambiguous rulebook specification (querying artifact)
\item Converting existing rulebooks for deck-building games into unambiguous programs written in
our grammar. Such programs can act as the `ground truth' when it comes to disputes over ambiguous
language used in the natural-language (English) version of rulebooks. It would thus be useful to
have a {\bf runtime system which can be queried} about the validity of certain actions given a game state.

% Use case #2 - prototyping deck-building games & running simulations (simulator artifact)
\item Prototyping combinations of cards to use in a specific instance of a deck-building game.
Given a particular game designed in our grammar, the game designer will want to determine the
fairness and variety of strategies available. Thus a useful runtime artifact is a tool for
running {\bf automated simulations} of actual gameplay.

% Use case #3 - Client-Server artifacts / boiler-plate code (networking artifact)
\item a {\bf client-server framework} that will enforce the rules of the game produced will provide a way to test out their games and various card definitions. The designer can easily distribute the game to playtesters as a specficiation consumed by a premade client and by deploying the generated game server. By encapsulating the cards and game rules as a server with an api, a variety of clients can be made to play against playtesters such as GUIs and other skins, or bot clients.

% Use case #3 - concrete specification language for use by an AI language (AI language artifact)
\item When human players analyze their gameplay strategies, they use various heuristics.
These heuristics only exist as a mental model in the human player's mind, meaning their
robustness cannot be effectively tested. We therefore hope to extend our DSL with probabilistic
language features for encoding such mental model heuristics into a concise machine-readable form.
The artifact would then be a {\bf playable AI}.

\end{enumerate}

\subsubsection{Existing Rulebook Syntax}
% This is a possible future goal for language features
\begin{verbatim}

[ Code for declaring the rules of a deck-building game goes here ]
\end{verbatim}

From this code we can generate a playable runtime system, with sufficient information
because the game designer has specified the exact mechanics of the game.

\subsubsection{Card Prototyping Syntax}
\begin{verbatim}

[ Code for declaring / creating new Dominion cards goes here ]
\end{verbatim}

From this code we also generate a playable runtime system with cards never before
used in a deck-building game. The information given is sufficient because the
game designer specifies the exact abilities of each card. In conjunction with the
rulebook syntax, we also know how the cards interact with the mechanics of the game.

\subsubsection{Client-Server API}
\begin{verbatim}
GET /
\end{verbatim}

\subsubsection{Computer Heuristic Syntax}
% Future goal
\begin{verbatim}

[ Code for declaring AI heuristics ]
\end{verbatim}

From this code we can create an interactive runtime system in which one or more of the players
of the deck-building game are not human. The heuristics defined are sufficient to create a
computer player because the heuristic programmer specifies exactly what the computer player
should do based on the current state of the game / run-time system.

\section{Haskell}
\label{sec:haskell}
% Question (7)
%   What existing Haskell libraries, if any, are you planning to use to support the
%   domain-specific aspects of your implementation? You don't need to mention
%   quasi-quotation, template Haskell, or parsing libraries.
\begin{itemize}
\item websockets
\end{itemize}
\section{Evaluation Criteria}
\label{sec:evaluation}
% Question (10)
%   Sketch a plan for evaluating your design. This plan should include measurable
%   goals that if achieved would lead you to conclude your design was a success.
%   (Example goals might be code-size reductions or measured performance overheads,
%   but many other goals are possible).


\begin{biblist}
  \bib{Vaccarino2008}{book}{
    title={Dominion},
    author={Vaccarino, Donald X.},
    author={Rio Grande Games},
    date={2008}
  }
\end{biblist}

%\balancecolumns

\end{document}
